%%%%%%%%%%%%%%%%%%%%%%%%%%%%%%%%%%%%%%%%%%%%%%%%%%%%%%%%%%%%%%%%%%%%%%%%%%%%%%%%%%%%%%%%%%%%%%%%%%%%%%
%
%   Filename    : chapter_2.tex 
%
%   Description : This file will contain your review of related works.
%                 
%%%%%%%%%%%%%%%%%%%%%%%%%%%%%%%%%%%%%%%%%%%%%%%%%%%%%%%%%%%%%%%%%%%%%%%%%%%%%%%%%%%%%%%%%%%%%%%%%%%%%%

\chapter{Related Works}
\label{sec:relatedworks}

\citet{Pennebaker1999} explored on their study how language use can reflect the personality style of someone. Using a computerized word-based text analysis program, the structure, validity, and reliability of the written language was examined. From there it was found that one's linguistic style is a consequential way to explore personality.

This was further explored in a study by \citet{Mairesse2007} in which the recognition of the Big Five Personality traits (Openness, Conscientiousness, Extroversion, Agreeableness, and Neuroticism) on both text and conversation was first experimented using both self rating and observer ratings of personality. From the experimentation done with the various models: classification, ranking, and regression; the ranking models performed best overall.

The PagkataoKo dataset analyzed in this paper has been used in the studies of undergraduates, graduates, and professionals at DLSU. These various studies primarily involved modeling Filipino users’ personality traits or Automatic Personality Recognition (APR) through their usage of the Instagram and X (formerly Twitter) social media sites, as was the purpose of the creation of the dataset. 

The dataset contains data on Filipino social media users’ social media activity on Instagram and Twitter. This includes features like post images and captions,  profile pictures, follower and following counts, and post count. It also includes the user’s BFI scores.

Among the undergraduate and graduate studies that pursued the analysis of this dataset, the majority of them utilized only the X data, focusing on natural language processing. Techniques such as word embedding models, topic modeling, feature extraction, and other models were used to predict the BFI personality traits of users. One used both X and Instagram, but only the text data was used. There also exists another study that analyzed the visual features of Instagram image data and used this to predict personality traits. Lastly, a published study also used text processing on the X data for the same purpose.

X was a more popular choice for analysis as the platform's main focus is that it is more text-based and the captions hold the most importance whereas on Instagram, it is more secondary. This leaves the Instagram portion of the dataset relatively unexplored compared to the X data, so there is little information about image processing in this dataset. However, this leaves open the potential for exploring images and their relation to personality, specifically the BFI scores, adding to the one research that was done on this topic. 

Filipinos use social media in many ways. They use it for connecting with other people, sharing information, speaking out, and optimizing productivity \citep{10.1007/978-3-031-61543-6_24}. However, even though they are aware of the presence of social media, \citet{Cruz_Jamias_2013} found that Filipino researchers rarely use social media to their advantage. This demonstrates that each demographic in the Philippines has a different level of social media use.

Social media has many effects in the Philippines. For instance, it plays a role in youth political participation. \citet{Ibardeloza2022-pz} found that social media helps the youth to be more exposed to radical involvement. It also has a positive effect on “peer interpersonal relationships” yet slightly negative on familial relationships \citep{Bristol2016TheDM}.

Recent studies have explored the potential of Instagram image data to infer users' personality traits. \citet{ferwerda2016} analyzed features such as hue, brightness, and saturation in users' photos, finding significant correlations between these visual elements and the Big Five personality traits. Their findings suggest that individuals' choices in photo appearance can reflect underlying personality characteristics. 

Further research by \citet{Reece2017-qw} utilized machine learning to examine Instagram photos for markers of depression. By analyzing color properties, metadata, and the presence of faces, their model successfully identified depressive indicators, highlighting the platform's potential for mental health screening. 

Additionally, a study by \citet{Harris2019-gq} investigated the accuracy of personality portrayal on Instagram by comparing observers' perceptions of account holders' personalities with the account holders' self-reported traits. The results indicated discrepancies, suggesting that while certain visual cues can convey personality aspects, they may also lead to misinterpretations. 

Collectively, these studies underscore the viability of leveraging Instagram image data to assess users' personality traits and mental health status, though they also caution against overreliance on visual cues due to potential misrepresentations.


%%This chapter provides a synthesis of past research, existing algorithms, and or state-of-the-art software that are related/similar to the thesis. It should not present detailed summaries of each related work, but rather present a cohesive comparison of different aspects of their work. At the end of each section and this chapter, it should be clear what research challenges and opportunities will be focused on for the proposal.

%%The sections can be about approaches, application areas, and categories of solutions that give readers an deep understanding of the current state of the field. 

%%Observe a consistent format when presenting each of the reviewed works. This must be selected in consultation with the prospective adviser.

%%\textcolor{red}{DO NOT FORGET to cite your references.} Related works can be discussed multiple times in different sections of this chapter, depending on what is being discussed or compared.


\begin{comment}
%
% IPR acknowledgement: the contents within this comment are from Ethel Ong's slides on RRL.
%
Guide on Writing your Related Works chapter
 
1. Identify the keywords with respect to your research
      One keyword = One document section
                Examples: 2.1 Story Generation Systems
			 2.2 Knowledge Representation

2.  Find references using these keywords

3.  For each of the references that you find,
        Check: Is it relevant to your research?
        Use their references to find more relevant works.

4. Identify a set of criteria for comparison.
       It will serve as a guide to help you focus on what to look for

5. Write a summary focusing on -
       What: A short description of the work
       How: A summary of the approach it utilized
       Findings: If applicable, provide the results
        Why: Relevance to your work

6. At the end of each section,  show a Table of Comparison of the related works 
   and your proposed project/system

\end{comment}
















