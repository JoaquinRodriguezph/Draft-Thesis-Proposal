%%%%%%%%%%%%%%%%%%%%%%%%%%%%%%%%%%%%%%%%%%%%%%%%%%%%%%%%%%%%%%%%%%%%%%%%%%%%%%%%%%%%%%%%%%%%%%%%%%%%%%
%
%   Filename    : abstract.tex 
%
%   Description : This file will contain your abstract.
%                 
%%%%%%%%%%%%%%%%%%%%%%%%%%%%%%%%%%%%%%%%%%%%%%%%%%%%%%%%%%%%%%%%%%%%%%%%%%%%%%%%%%%%%%%%%%%%%%%%%%%%%%

\begin{abstract}
Consists of 150 to 250 words in a \textcolor{blue}{\textbf{single paragraph}}, that provides the reader with a summary of your research/study and what you intend to do. It should be informative enough to serve as a substitute for reading the thesis document itself. The Abstract should contain the following sentences:
\begin{itemize}
   \item \textbf{Motivation}. One to two sentence(s) describing the research domain or problem area.
   \item \textbf{Problem}. One to two sentence(s) describing the research challenge or opportunity to be addressed in your study
   \item \textbf{Contribution}. One to two sentence(s) describing your intended contribution or how you plan to address the research question 
   \item \textbf{Methodology}. One to two sentence(s) summarizing the approach or the manner in which you will carry out your study
   \item \textbf{Significant Results or Findings}. One to two sentence(s) describing your findings and result (This can be skipped during proposal stage) 
\end{itemize}


%
%  Do not put citations or quotes in the abract.
%

Keywords can be found at \url{http://www.acm.org/about/class/class/2012?pageIndex=0}.  Click the 
link ``HTML'' in the paragraph that starts with ''The \textbf{full CCS classification tree}...''.

\begin{flushleft}
\begin{tabular}{lp{4.25in}}
\hspace{-0.5em}\textbf{Keywords:}\hspace{0.25em} & Keyword 1, keyword 2, keyword 3, keyword 4, etc. \textit{minimum of three relevant and descriptive keywords}\\
\end{tabular}
\end{flushleft}
\end{abstract}
